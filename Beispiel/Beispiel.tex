\documentclass[table,xcdraw,12pt]{scrartcl}  

\usepackage[german]{babel}

\usepackage{amsmath}
\usepackage{epigraph}
\usepackage{booktabs}
\usepackage{xcolor}
\usepackage{fancyhdr}
\usepackage{tikz}
\usepackage[utf8]{inputenc}
\usepackage{titling,lipsum}

\renewcommand\epigraphflush{flushright}
\renewcommand\epigraphsize{\normalsize}
\setlength\epigraphwidth{0.7\textwidth}
\setlength\parindent{0pt}

% Section numbering.  
% Here again is a variable you can specify on the commandline
% `markdown2pdf my.txt --number-sections --xetex --template=/wherever/this/is -o my.pdf`
%\setcounter{secnumdepth}{0}


\definecolor{titlepagecolor}{cmyk}{1,.60,0,.40}

\DeclareFixedFont{\titlefont}{T1}{ppl}{b}{it}{0.5in}


% Footnotes

% Lists formatting: 
% note sure what 'fancy enums' are; something to do with lists, 
% as the further comment suggests: 


% Table formatting: 
% What if you make a table? -- Pandoc knows, of course, and 
% then declares that its  variable `table` is True and 
% imports a table package suitable to its pleasantly simple tables. 
% Needless to say infinitely   complicated tables are possible in 
% LaTeX with suitable packages. We are spared the temptation:



% Subscripts:
% Pandoc remembers whether you used subscripts, assigning True to 
% its `subscript` variable 
% It then needs to adopt a default with an incantation like this:


% Web-style links:
\usepackage[breaklinks=true]{hyperref}
\hypersetup{colorlinks,%
citecolor=blue,%
filecolor=blue,%
linkcolor=blue,%
urlcolor=blue}


% Images. 
% In ye olde LaTeX one could only import a limited range of image
% types, e.g. the forgotten .eps files.  Or else one simply drew the image with suitable
% commands and drawing packages.  Today we want to import .jpg files we make with 
% our smart phones or whatever:



% Section numbering.  
% Here again is a variable you can specify on the commandline
% `markdown2pdf my.txt --number-sections --xetex --template=/wherever/this/is -o my.pdf`
\setcounter{secnumdepth}{0}

% Footnotes: 
% Wait, didn't we already discuss the crisis of code in footnotes?  
% Evidently the order of unfolding of macros required that
% we import a package to deal with them earlier
% and issue a command it defines now. (Or maybe that's not the reason;
% very often the order does matter as the insane system of macro expansion
% must take place by stages.)

% Other stuff you specify on the command line:
% You can include stuff for the header from a file specified on the command line;
% I've never done this, but that stuff will go here:

% Title, authors, date.
% If you specified title authors and date at the start of 
% your pandoc-markdown file, pandoc knows the 'values' of the
% variables: title authors date and fills them in.

\title{Lastenheft}
\author{}
\date{12.05.2018}


% The following code is borrowed from: https://tex.stackexchange.com/a/86310/10898

\newcommand\titlepagedecoration{%
\begin{tikzpicture}[remember picture,overlay,shorten >= -10pt]

\coordinate (aux1) at ([yshift=-15pt]current page.north east);
\coordinate (aux2) at ([yshift=-410pt]current page.north east);
\coordinate (aux3) at ([xshift=-4.5cm]current page.north east);
\coordinate (aux4) at ([yshift=-150pt]current page.north east);

\begin{scope}[titlepagecolor!40,line width=12pt,rounded corners=12pt]
\draw
  (aux1) -- coordinate (a)
  ++(225:5) --
  ++(-45:5.1) coordinate (b);
\draw[shorten <= -10pt]
  (aux3) --
  (a) --
  (aux1);
\draw[opacity=0.6,titlepagecolor,shorten <= -10pt]
  (b) --
  ++(225:2.2) --
  ++(-45:2.2);
\end{scope}
\draw[titlepagecolor,line width=8pt,rounded corners=8pt,shorten <= -10pt]
  (aux4) --
  ++(225:0.8) --
  ++(-45:0.8);
\begin{scope}[titlepagecolor!70,line width=6pt,rounded corners=8pt]
\draw[shorten <= -10pt]
  (aux2) --
  ++(225:3) coordinate[pos=0.45] (c) --
  ++(-45:3.1);
\draw
  (aux2) --
  (c) --
  ++(135:2.5) --
  ++(45:2.5) --
  ++(-45:2.5) coordinate[pos=0.3] (d);   
\draw 
  (d) -- +(45:1);
\end{scope}
\end{tikzpicture}%
}


% At last: 
% The document itself!:

% After filling in all these blanks above, or erasing them 
% where they are not needed, Pandoc has finished writing the 
% famous LaTeX *preamble* for your document.
% Now comes the all-important command \begin{document}
% which as you can see, will be paired with an \end{document} at the end.
% Pandoc knows whether you have a title, and has already
% specified what it is; if so, it demands that the title be rendered.  
% Pandoc knows whether you want a table of contents, you
% specify this on the command line.
% Then, after fiddling with alignments, there comes the real
% business: pandoc slaps its rendering of your text in the place of
% the variable `body`
% It then concludes the document it has been writing. 

\begin{document}

% ----------------------------------------------------------------------------------
% Titelseite
% ----------------------------------------------------------------------------------

\begin{titlingpage}
    \noindent
    \Huge \textbf{\uppercase{Lastenheft}}

    \vspace{1.5cm}
    \hfill\begin{minipage}{\dimexpr\textwidth-1cm}
    \Large 
    \textbf{Projekttitel:} Chaos Conquerer\\
    \textbf{Version:} 1.0 \\
    \textbf{Datum:} 12.05.2018 
    \end{minipage}

    \null\vfill

    \vspace*{1cm}

        \large
    \textcolor{gray}{\textbf{\uppercase{Dokumentenversion}}}

    \small
    \begin{table}[!htb]
    \centering
    \resizebox{\textwidth}{!}{\begin{tabular}{@{}llll@{}}
    \toprule
    Version & Datum       & Autor              & Bemerkungen \\ \midrule
            0.1   & 10.05.2018 & Sebastian Preisner & Entwurf des
Dokuments   \\ \bottomrule
            0.5   & 12.05.2018 & Aud
Peterson & Rechtschreibkorrektur   \\ \bottomrule
            1.0   & 12.05.2018 & Sebastian Preisner & Finalisierung des
Lastenhefts   \\ \bottomrule
        \end{tabular}}
    \end{table}
    

    \titlepagedecoration
\end{titlingpage}





\hypertarget{beispiel-lastenheft}{%
\section{Beispiel Lastenheft}\label{beispiel-lastenheft}}

Dies ist eine Beispieldatei für ein Lastenheft mit der Pandocvorlage.

%

\end{document}
